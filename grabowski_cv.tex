% !TeX spellcheck = en_GB
% !TeX program = pdflatex
%
% LetzCV-sleek 1.0 LaTeX template
% Author: Andreï V. Kostyrka, University of Luxembourg
%
% This template fills the gap in the available variety of templates
% by proposing something that is not a custom class, not using any
% hard-coded settings deeply hidden in style files, and provides
% a handful of custom command definitions that are as transparent as it gets.
% Developed at the University of Luxembourg.
%
% *NOTHING IS HARCODED, and never should be.*
%
% Target audience: applicants in the IT industry, or business in general
%
% The main strength of this template is, it explicitly showcases how
% to break the flow of text to achieve the most flexible right alignment
% of dates for multiple configurations.

\documentclass[11pt, a4paper]{article} 

\usepackage[T1]{fontenc}     % We are using pdfLaTeX,
\usepackage[utf8]{inputenc}  % hence this preparation
\usepackage[british]{babel}  
\usepackage[left = 0mm, right = 0mm, top = 0mm, bottom = 0mm]{geometry}
\usepackage[stretch = 25, shrink = 25]{microtype}  
\usepackage{graphicx}        % To insert pictures
\usepackage{xcolor}          % To add colour to the document
\usepackage{marvosym}        % Provides icons for the contact details

\usepackage{fontawesome5}	% Cause we need some awesome icons
\usepackage{enumitem}        % To redefine spacing in lists
\setlist{parsep = 0pt, topsep = 0pt, partopsep = 1pt, itemsep = 1pt, leftmargin = 6mm}

\usepackage{FiraSans}        % Change this to use any font, but keep it simple
\renewcommand{\familydefault}{\sfdefault}

\definecolor{cvblue}{HTML}{043540}

%%%%%%% USER COMMAND DEFINITIONS %%%%%%%%%%%%%%%%%%%%%%%%%%%
% These are the real workhorses of this template
\newcommand{\dates}[1]{\hfill\mbox{\textbf{#1}}} % Bold stuff that doesn’t got broken into lines
\newcommand{\is}{\par\vskip.5ex plus .4ex} % Item spacing
\newcommand{\smaller}[1]{{\small$\diamond$\ #1}}
\newcommand{\headleft}[1]{\vspace*{3ex}\textsc{\textbf{#1}}\par%
    \vspace*{-2ex}\hrulefill\par\vspace*{0.7ex}}
\newcommand{\headright}[1]{\vspace*{2.0ex}\textsc{\Large\color{cvblue}#1}\par%
     \vspace*{-2ex}{\color{cvblue}\hrulefill}\par}
%%%%%%%%%%%%%%%%%%%%%%%%%%%%%%%%%%%%%%%%%%%%%%%%%%%%%%%%%%%%

\usepackage[colorlinks = true, urlcolor = white, linkcolor = white]{hyperref}

\begin{document}

% Style definitions -- killing the unnecessary space and adding the skips explicitly
\setlength{\topskip}{0pt}
\setlength{\parindent}{0pt}
\setlength{\parskip}{0pt}
\setlength{\fboxsep}{0pt}
\pagestyle{empty}
\raggedbottom

\begin{minipage}[t]{0.33\textwidth} %% Left column -- outer definition

\vspace{-.2ex} % Eliminates the small gap
\colorbox{cvblue!90}{\color{white}\hfill  %% LEFT BOX
\kern0.09\textwidth\relax% Left margin provided explicitly
\begin{minipage}[t][298mm][t]{0.84\textwidth}
\raggedright
\vspace*{1.5ex}

\vspace{3ex}
\begin{center}
    \Huge \textbf{ALEXANDER}\vspace{0.5ex}
    \textbf{GRABOWSKI} \normalsize 
\end{center}

% Centering without extra vertical spacing
\vspace*{-2.5ex} 

\headleft{Contact}
\faEnvelope\ \href{mailto:xander@xangrab.com}{xander@xangrab.com} \\[0.4ex]

\faGlobe\ \href{https://xangrab.com/}{https://xangrab.com/} \\[0.4ex]

\faGithub\ \href{https://github.com/XanGrab/}{https://github.com/XanGrab/} \\[0.4ex]

\faItchIo\ \href{https://xangrab.itch.io/}{https://xangrab.itch.io/} \\[0.4ex]


\headleft{Profile}
I have been professionally developing games for the last two years, and recreationally my entire life. I am proficient in a diverse range of full-stack toolchains targeting game and web development. I can frequently be found in international fan games, hackathons, and game jams, expressing my passion for applied learning games and FOSS. I developed a sense of leadership early, achieving the rank of Eagle Scout, and continued developing my leadership skills guiding competitive e-sports teams to compete at the national level. \textbf{I am a lifelong, self-driven learner and problem solver.}

\headleft{Skills}
\small{
\begin{itemize}
\item Game Engines
\setlist{label=$\circ$, parsep = 0pt, topsep = 0pt, partopsep = 1pt, itemsep = 1pt, leftmargin = 6mm}
\begin{itemize}
	\item Unity
	\item Godot
	\item Phaser
\end{itemize}
\item Graphics Libraries
\begin{itemize}
	\item Pixi.js
	\item HLSL (Unity)
	\item GLSL (Three.js)
\end{itemize}
\item Programming Languages
\begin{itemize}
	\item C++
	\item C\#
	\item TypeScript
\end{itemize}
\item Web Development Tools
\begin{itemize}
	\item Node.js
	\item Vite
	\item Webpack
\end{itemize}
\item DevOps \& Services
\begin{itemize}
	\item Github Actions
	\item Docker
	\item Firebase BaaS
\end{itemize}
\item Digital Illustration \& Wireframing
\begin{itemize}
	\item Figma
	\item Illustator
	\item Photoshop
\end{itemize}
\end{itemize}
}

\end{minipage}%
\kern0.09\textwidth\relax%%Right margin provided explicitly to stretch the colourbox
}
\end{minipage}% Right column
\hskip2.5em% Left margin for the white area
\begin{minipage}[t]{0.56\textwidth}
\setlength{\parskip}{0.8ex}% Adds spaces between paragraphs; use \\ to add new lines without this space. Shrink this amount to fit more data vertically

\vspace{3ex} %offset the right collumn from the top

\hypersetup{
	urlcolor = black, 
	linkcolor = black
}
\headright{Experience}

\textbf{Gameplay Programmer} at \textit{Field Day Learning Games}  \dates{2022.01--2023.05} \\
\url{https://fielddaylab.wisc.edu/} \\
\small{Developer and junior designer on multiple projects including \textit{Wake: Tales from the Aqualab}, client-side API for \textit{Open Game Data}, and samller browser-based HTML5 games}
\is 

\textbf{Head Programming Instructor} at \textit{Code Ninjas}  \dates{2020.10--2022.12} \\
\url{https://codeninjas.com/} \\
\small{Developed programming curriculum for kids of various backgrounds in the K-12 age range spanning  JavaScript, MakeCode Arcade, Roblox Studio, and the Unity game engine}
\is 

\textbf{Youth Esports Coach} at \textit{XP League}  \dates{2020.10--2022.12} \\
\url{https://www.xpleague.com/} \\
\small{Certified youth eSports coach through the Positive Coaching Alliance. Led the Sun Prairie league's Fortnite team to compete at the North American Finals 2022}
\is

\headright{Projects}

\textsc{MOTHER².} \\ \textit{Unattached} \dates{2021--pres.} \\
\url{https://www.youtube.com/watch?v=zWeF1jRstLk} \\
\small{Released in 1994 as MOTHER 2 in Japan and as EarthBound in the west, MOTHER² is a full-length ground-up reimagining of the game by fans for fans.}
\is

\textsc{Wake: Tales from the Aqualab.} \\ \textit{Field Day Learning Games} \dates{2022--2023} \\
\url{https://fielddaylab.wisc.edu/play/wake/} \\
\small{An oceanography game targeting Chromebooks in middle school classrooms helping students learn scientific experimentation, modeling, and argumentation.}\\
\smaller{Designed pivotal story and level layout elements of the introductory sequence for the game}\\
\smaller{Brought gameplay systems to life in Unity, implementing user interfaces for the game's Shop, Observation Tanks, and Modeling}\\
\smaller{Used in-house scripting tools to implement over seventy-five percent of the quests in the game}
\is

\textsc{Unannounced HTML5 Browser Games.} \\ \textit{Field Day Learning Games} \dates{2023} \\
\url{https://github.com/opengamedata/opengamedata-js-log/} \\
\small{Point-and-click, browser-based games targeting mobile tablets and other handheld devices.}\\
\smaller{Built data analytics and logging tools for real-time data collection with Firebase in collaboration with the Open Game Data project}\\
\smaller{Developed the game’s dialogue system building text-parsing tooling which extended scripts written in YarnSpinner}\\
\smaller{Adapted the game's graphics to make use of both WebGL and Canvas rendering styles to meet different tablet and mobile limitations}
\is

\headright{Education}
\textsc{User Experience Design Capstone Certificate} \\
\textit{University of Wisconsin\textemdash Madison} \dates{2023--pres.}
\is

\textsc{Bachelor of Arts in Computer Science} \\
\textit{University of Wisconsin\textemdash Madison} \dates{2019--2023}
\begin{itemize}
    \item Certification (Minor) in Game Design
    \item Certification in Digital Art
    \item Certification in Professional Japanese Communication
\end{itemize}

\end{minipage}
\end{document}
